\chapter{Methodology}
\label{chap:methodology}

This chapter describes the methodology used to compare and rank the studied cross-platform tools. The methodology follows the suggested 6-step approach presented in \cite{Jadhav:2011} (see chapter \ref{sec:sw-selection}). The six steps are:

\begin{enumerate}
    \item Define selection criteria
    \item Identify potential candidates
    \item List selected alternatives
    \item Define evaluation criteria
    \item Evaluate selected alternatives
    \item Select the most suitable alternative
\end{enumerate}

Every step will be further expanded in the following sections.

\section{Define selection criteria}
\label{sec:selection-criteria}

In this first step, the selection criteria for the tool are recorded. These requirements will be used later (in step 3) to filter a list of potential candidates. 

As the research presented in this thesis is conducted on behalf of CapGemini, the selection criteria are determined by their consultants and were recorded during the kick-off meeting on October 15, 2012. In order to qualify as a viable cross-platform tools, it has to meet the following requirements:

\begin{itemize}
    \item \textbf{It \emph{must} produce ``native'' Android \emph{and} iOS applications.} CapGemini focusses mainly on Android and iOS because their clients mainly focus on these platforms. Support for other platforms is desirable but not not a necessity.
    \item \textbf{It \emph{must} be able to produce \emph{both} tablet and smartphone applications, preferably from the same codebase.} Some clients want tablet applications, some clients want smartphone apps, some clients want both.
\end{itemize}

This list of essential requirements is extended with additional requirements. These are not essential as they can be circumvented in some way though they will generally result in higher productivity.
    
\begin{itemize}    
    \item \textbf{It \emph{should} be usable to create enterprise applications with.} CapGemini specializes in the development of data-driven enterprise applications. Such applications usually contain a lot of forms and don't require high performance graphics (like for instance in 3D games). Even though it is possible to develop an enterprise application on top of a 3D engine, it will probably not result in good productivity.
    \item \textbf{It \emph{should} have a certain degree of maturity} Ideally, the tool should be maintained for as long applications are created with it (and maintained). 
    \item \textbf{It \emph{should} have good support, provided by either the vendor or by the community.} In case of a problem, there should be a way to get support.
\end{itemize}

\section{Identify potential candidates}

In this stage, the evaluator tries to identify as much potential candidates as possible. These candidates do not necessarily have to meet the requirements from the previous stage as this is merely a discovery phase. The result of this stage will be a list of potential candidates.

Discovery of cross-platform tools has already been done extensively by VisionMobile. The latest cross-platform tools report \cite{VMCPT:2012} contains a list of 100 cross-platform tools they tracked as part of their research. Because additional internet searches did not reveal new tools, this list is used as output of this stage.

\section{List selected alternatives}

In this phase, the candidates obtained from the previous stage are filtered with the selection criteria from the first stage. The result of this stage is a list of alternatives worth investigating.

The list of tools from the previous stage contains a plethora of tools. Using the requirements from stage 1, a large number of tools can be left out:

\begin{itemize}
    \item tools that do not produce native applications for Android and iOS;
    \item tools that do not produce tablet and smartphone applications;
    \item special-purpose tools that are not well suited for the intended use, e.g. specialized 3D engines;
    \item tools with an uncertain future, e.g. Flash-based systems or cutting-edge tools;
    \item tools that do not offer good support. 
\end{itemize}

From 100 potential candidates, 7 tools listed in \tref{table:tools} were selected.

\begin{table}[h!]
    \begin{center}
        \begin{tabular}{lcc}
            \hline
            Name & Architecture & Type \\
            \hline 
            Apache Cordova & Hybrid & Open Source \\
            Appcelerator Titanium & Interpreted & Open Source \\
            Motorola Rhodes & Interpreted & Open Source \\
            Trigger.io & Hybrid & Commercial \\
            MoSync & Cross-Compiled + Hybrid & Open Source \\
            Kony & Hybrid & Commercial \\
            Xamarin & Cross-compiled & Commercial \\
            \hline
        \end{tabular}
        \caption{The remaining tools after application of the selection criteria.}
        \label{table:tools}
    \end{center}
\end{table}

This thesis will compare two of these tools with each other and  with the native development kits for both Android and iOS. The selected tools are Apache Cordova (formerly known as PhoneGap) and Motorola Rhodes. 

Apache Cordova was chosen because of its popularity among developers and Motorola Rhodes was chosen because it focusses on enterprise applications. 

Appcelerator Titanium and Xamarin were not selected because they are evaluated by another student with the same research topic. 

Trigger.io was not selected because it was too similar to Apache Cordova at first sight: same architecture, same languages, \ldots

\TODO{From what I read now, I should have selected MoSync, why didn't I? Don't really have a reason for it except maybe that I didn't research it quite well. Was also not prioritized by CapGemini but might have been my influence...}

Kony --- even though apparently using the the same architecture as Apache Cordova --- targets enterprise applications, which made it a very good rival for Motorola Rhodes. However, there was no free trial available for Kony, which resulted in the selection of Motorola Rhodes.

Apache Cordova and Motorola Rhodes are discussed in more detail in chapter \ref{chap:tools}.

\section{Define evaluation criteria}

This is probably the most important stage of all. Setting up good grounds for comparison is paramount. The criteria are therefore carefully recorded from interviews with CapGemini and literature.

\subsection{CapGemini Interviews}

A finished application is the result of a cooperation of different people with various roles. Every person experiences mobile application development in another way. In order to gain a better understanding of these experiences, three interviews with CapGemini employees were planned: one interview with a developer, one interview with a mobile architect and one interview with a sales representative. 

The essence of these interviews is to reveal important criteria that should be considered during the evaluation of the cross-platform tools.

The following sections summarize the findings of these interviews

\subsubsection{Developer}

\paragraph{Development environment}

One of the relevant topics for application developers is the development environment since they will spend most of their time in it. Typically, the development environment is built on top of a Windows operating system and the Eclipse\footnote{\url{http://eclipse.org}} IDE. This IDE is preferred because it is fast, efficient and highly customizable, which are desirable qualities for a good IDE. Also, most of the documentation is explicitly written for Eclipse. 

\paragraph{Development cycle}

Development happens locally and changes are deployed in an \emph{unstable}  environment. When features are ready to be tested, they are promoted and pushed to a \emph{staging} environment, where they will be tested by for instance a number of end users. When the features are tested and ready to ship, a new production release is planned. In case mobile devices are involved, the application is tested on the device as much as possible.

The code is also checked overnight on a continuous integration server, which runs all unit tests together with a number of other useful tools like FindBugs\footnote{FindBugs uses static analysis to find common bugs, \url{http://findbugs.sourceforge.net}.}. Every morning, a detailed report of the state of the code is generated.

\paragraph{Requirements for cross-platform tools}

The requirements for cross-platform tools are that it (1) should work properly, (2) should not be too complex, (3) has at least decent performance (preferred over ease of use) and (4) that it is customizable graphically. Most clients have specific styling requirements.

\paragraph{Attitude towards new languages and tools} 

Learning new tools and languages is not considered problematic as long as it is worth investing in and/or requested by clients. Typically, one or more developers familiarize themselves with the tool/language and teach the other developers during a training session.

\subsubsection{Mobile architect}

\paragraph{Focal points in mobile architectures} 

Mobile architectures consist mainly of two major components: a front-end component and a back-end component. On the side of the front-end (the mobile application), the most important factor is the degree of integration with the device.

On the side of the back-end, the most important factors are the services the application needs to integrate with. An additional server, called the \emph{mobile orchestrator} or \emph{mobile hub}, will handle this integration and expose it in a single interface to the mobile application. This protects the back-end from the increased number of requests and allows for caching and other integrations like targeted advertisement campaigns.

\paragraph{Requirements for cross-platform tools}

There are a number of requirements for cross-platform tools:
\begin{itemize}
    \item \textbf{Adoption} How many developers are using this tool? The number of users is mostly a good measure for the quality of the tool. A large user base is also beneficial when facing problems because it is more likely to get answers.
    \item \textbf{Future proof} Will the tool survive? Is there enough financial backing or community effort? It is not very wise to start a project that relies on a tool that has an unclear future. 
    \item \textbf{Documentation} Good documentation is priceless as it can save numerous hours of digging through code. 
    \item \textbf{Support} In case of a problem, is there an official channel to open a ticket? This could be either paid support or good community support. Either way, it should solve the problem. Delay due to unsolvable problems is unacceptable.
    \item \textbf{Customizability} Clients have specific requirements with regard to user interfaces and the tool should make this possible. Also, can the tool be expanded with plugins? A plugin architecture can contribute substantially to the customizability of the tool.
    \item \textbf{Supported devices} Which devices and runtimes does the tool support? In a B2B\footnote{Business-to-business, or commerce between businesses} context, devices are mostly controlled by an organization and the tool does not need to support many devices. In a B2C\footnote{Business-to-consumer, or commerce between a business and consumers} context on the other hand, the organization cannot control the devices and consequently, the tool should support a lot more devices. Does the tool cover most or all of the used devices in this case?
    \item \textbf{Performance} The benefits of cross-platform development must justify the costs in terms of performance. If it took virtually no time to develop the application but it is slow, the application is useless. 
    \item \textbf{Tools} Does this cross-platform tool allow to reuse the development tools that are already in use? Most people are more productive with the tools they are familiar with.
\end{itemize}

\paragraph{Usability} 

Usability is a big concern when developing cross-platform. Users are accustomed to the native look \& feel and wish to see new applications with similar styling. Even HTML5 applications are designed to mimic the native look \& feel of the device. It is therefore hard to write applications with a single code base, even though this is desirable. Only when the application has a custom user interface that is in no way related to the native look \& feel of the devices, the user interface can be reused across devices.

\subsubsection{Sales representative}

\TODO{}

\subsection{Literature}

Another source from where criteria could be identified is literature.

Similar research has already been performed by VisionMobile\footnote{VisionMobile is an ecosystems analyst firm, \url{http://www.visionmobile.com/}.}. Their report on cross-platform tools \cite{VMCPT:2012} contains the results from a survey among 2406 developers across 91 countries.

Two of the questions that were asked are ``What are the key reasons for tool selection?'' and ``What are the key reasons to drop a tool?''. 

\begin{itemize}
    \item It supports the platforms I am targeting, 61\%
    \item Allows me to use my existing skills, 43\%
    \item Low cost or free, 40\%
    \item Rapid development process, 33\%
    \item Easy learning curve, 23\%
    \item Rich UI capabilities, 19\%
    \item Access to device or hardware APIs, 10\%
    \item High Performance / low runtime overhead, 9\%
    \item Well suited for games development, 8\%
    \item Good vendor support and services, 8\%
\end{itemize}

\TODO{Define evaluation criteria from (1) VisionMobile Report and (2) Interviews with CapGemini + PoC}

\subsection{Summary of evaluation criteria}

\section{Evaluate selected alternatives}

\TODO{Using AHP, compare alternatives. Ask Jan to make 1 score table, derive 1 score table from VisionMobile report. Do the comparison twice or give both scoring tables adequate weights}

In this stage, the actual evaluation of the selected tools takes place. The alternatives are compared in pairs, with respect to the criteria recorded in the previous stage. In order to gain the necessary experience that is needed to formulate an accurate judgement, every tool is used to create a proof of concept application, or parts thereof.

The proof of concept application is discussed in section \ref{sec:poc}, the evaluation method is discussed in section \ref{sec:evaluation-method}. The actual evaluation of the tools is described in chapter \ref{chap:evaluation}.

\subsection{Proof-of-Concept application}
\label{sec:poc}

The proof-of-concept application is a rather small but typical enterprise application. It is not typical in the sense that it ``looks'' like the average application. Instead, it is rather typical in the sense that is contains the most requested features. This helps to ensure that the selected tools are thoroughly tested with regard to these essential features. This section describes these features in detail. The requirements documentation is available in Appendix \ref{app:poc}.

\subsubsection{Context \& scenario}

Employees of certain companies occasionally have to make costs for which they would like to be reimbursed. The process for this reimbursement typically involves keeping books, filling out forms and a lot of waiting while a superior deals with the request. Needless to say, there is a great potential for a mobile application here.

The application is designed to do just that. Employees can group a number of invoices into one request, provide evidence for the costs in the form of pictures, sign the document on a phone or tablet and send it to the backend. From there, the request is forwarded to a qualified person that will review the request and deal with it. 

\subsubsection{Typical functional elements}

The proof-of-concept application contains a number of features that are typically required in any application.

\begin{itemize}
    \item \textbf{UI elements} Most applications are useless without a user interface. This application incorporates a number of frequently used UI elements.
    \begin{itemize}
        \item \textbf{Form elements} Virtually all input is captured with ``forms''. The form elements should support read-write and read-only mode. This application uses different types of form elements to represent different kinds of data. 
        \begin{itemize}
            \item \textbf{Text} For inputting arbitrary text.
            \item \textbf{Number} For inputting numbers.
            \item \textbf{Email} For inputting email addresses.
            \item \textbf{Password} For inputting passwords, the content of the input field not shown as characters but as symbols. 
            \item \textbf{Drop-down} For selecting an item from a list.
            \item \textbf{Radio button} Also for selecting an item from a list.
            \item \textbf{Toggle switch} To toggle a state on a property, e.g. an on/off switch.
        \end{itemize}
        \item \textbf{Button} 
        \item \textbf{Tab bar}  
        \item \textbf{Spinner} A spinning wheel that is displayed when the user is supposed to wait while the applications handles some request.
    \end{itemize}
    \item \textbf{UI modes} The application must support multiple screen modes
    \begin{itemize}
        \item \textbf{Tablet UI} The display mode used on tablets. The layout typically consists of a narrow column on the left and a wide column on the right (also known as the master--detail interface).
        \item \textbf{Phone UI} The display mode used on smartphones. Nearly the same layout but shows less detail master and detail views are decoupled in separate views.
    \end{itemize}
    \item \textbf{serialization} In order to communicate with the backend,  data must be serialized using various formats. This application makes use of XML, JSON and plaintext.
    \item \textbf{Input validation}
    \begin{itemize}
        \item \textbf{Data type validation} 
        \item \textbf{Custom validation} When some  
    \end{itemize}
    \item \textbf{Sorting}
    \item \textbf{Offline mode}
    \item \textbf{Local Storage}
    \item \textbf{Session management}
\end{itemize}









\subsection{Evaluation methodology}
\label{sec:evaluation-method}


\section{Select the most suitable alternative}

\TODO{Describe the advantages and disadvantages of the three candidates. Make a decision if only these three candidates are available }












